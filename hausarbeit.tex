\documentclass[11pt,a4paper,toc=bibliography,toc=listof,titlepage=firstiscover]{scrreprt}
\usepackage[utf8]{inputenc}
\usepackage{array}
\usepackage{blindtext}
\usepackage{graphicx}
\usepackage{subcaption}
\usepackage[T1]{fontenc}
\usepackage{setspace}
\usepackage{pdfsync}
\usepackage{blindtext}
\usepackage[ngerman]{babel}
\usepackage{booktabs}
\usepackage{xcolor}
\usepackage{multirow}
\usepackage{makecell}
%Schriftarten hier definieren
\usepackage{lmodern}
%\usepackage{tgheros}
%\renewcommand*\familydefault{\sfdefault}
%\usepackage{tgpagella}
\usepackage[babel,german=guillemets]{csquotes}
\usepackage[style=apa]{biblatex}
\usepackage{hyperref}
\usepackage{csquotes}
\addbibresource{res/literatur.bib}
\hypersetup{colorlinks=true, linkcolor=black, urlcolor=blue, citecolor=blue}
%Abstand der Einträge im Literaturverzeichnis als em
\setlength\bibitemsep{0.75em}
%Kein Einzug bei einem neuen Absatz
\setlength{\parindent}{0mm}
%Inhaltsverzeichnis ohne eigene Seitenzahl
\AtBeginDocument{\addtocontents{toc}{\protect\thispagestyle{empty}}} 
\usepackage[left=3cm,right=2.5cm,top=2.5cm,bottom=2cm]{geometry}

%Daten des Deckblatts
  \titlehead
  {
    {
      \begin{center}
        \vspace{2.5cm}
        \includegraphics[scale=0.3]{res/nak-logo.png}
      \end{center}
    }
  	%M.Sc. Angewandte Informatik
  }
  \subject{Hausarbeit}
  \title{Einflussfaktoren auf die Schulleistungen im Rahmen der Oberschule}
    \subtitle{MADS2100 Reporting und Visualisierung 23oB}
    \author{Leon Henne}
  \date{\small{Köln, den \today}}
  \publishers{Betreut durch Dr. Robert Stahlbock}
\begin{document}
%Titelseite erzeugen
\maketitle
%Der Nachfolgende Text wird 1,5-zeilig gesetzt
\onehalfspacing
%Römische Ziffern für Abbildungsverzeichnis
\pagenumbering{Roman}
\setcounter{page}{0}
%Inhaltsverzeichnis erzeugen
\tableofcontents
\clearpage
%Abbildungs und Tabellenverzeichnis
\listoftables
\listoffigures
\clearpage
\pagenumbering{arabic}
\setcounter{page}{1}
%Datei aus dem Ordner /chapter/ einbinden. So kann man mehrere Textdateien zu einem großen Dokument zusammenfügen.
\setcounter{chapter}{0}

\chapter[Problemstellung]{Problemstellung}

Der für jede Nation erstrebenswerte langfristige ökonomische Fortschritt wird unter anderem stark durch das vorherrschende Bildungsniveau beeinflusst (\cite[S. 1]{Cortez2008UsingDM}). 
Um dieses sich ergebende Bildungsniveau stärker zu durchleuchten und final zu verbessern, wird zur Unterstützung der Schüler und Lehrkräfte die Modellierung von Schulleistungen eingesetzt. \cite[S. 1]{Cortez2008UsingDM} 
So kann eine zeitabhängige Vorhersage der Leistungen, lernschwächere Schüler detektieren und damit Lehrkräfte frühzeitig befähigen, mit entsprechenden Maßnahmen in den Lernprozess einzugreifen. \cite[S. 2]{Namoun.2021} 
Verstärkt wurde dieser Bedarf durch die in der Vergangenheit eingetretenen Covid-Pandemie, und den damit verbundenen Schulschließungen, welche für neue erhebliche Herausforderungen sorgten. \cite[S. 2]{Clark.2021} 
Durch \cite[S. 13]{Clark.2021} konnte hierzu aufgezeigt werden, welchen positiven Effekt digitale Lernunterstützungen auf die Schülergruppen ausmachten.
Aus der Arbeit von \cite[S. 9]{Namoun.2021} geht jedoch hervor, dass bereits seit 2017 erneut das Interesse anstieg hinsichtlich der Modellierung von Lernergebnissen.
Seitdem besteht besonders ein Fokus auf die Untersuchung des Bildungsniveaus von Bachelorstudiengängen, sodass die Untersuchung weiterführender Schulen lediglich ein Anteil von in etwa 12\% besitzt. \cite[S. 11]{Namoun.2021} 
Die in den letzten Jahren erforschte Modellierung von Studierendenergebnissen lässt häufig unbeachtet, wie einzelne Faktoren innerhalb der maschinellen Lernmethoden zu den Vorhersagen führen. \cite[S. 19]{Namoun.2021}
Die Gesamtheit dieser aktuellen Gegebenheiten motiviert die nachfolgende Untersuchung des gewählten Datensatzes anhand der daran abgeleiteten Forschungsfragen.
\chapter[Zielsetzung]{Zielsetzung}

Inhalt dieses Kapitels ist zunächst die Erläuterung des gewählten Datensatzes.
Zum Anderen werden daraus abgeleitete Forschungsfragen formuliert, welche im Rahmen dieser Arbeit durch die Entwicklung und Analyse von Visualisierungen untersucht werden. 

\section[Untersuchter Datensatz]{Datensatz}

Der in dieser Arbeit betrachtete Datensatz entstammt der Arbeit von \cite[]{Cortez2008UsingDM}.
Motiviert wurde diese Forschung durch Statistiken, welche Portugal im europäischen Vergleich als deutlich unterdurchschnittlich klassifizierten, aufgrund von hohen Durchfallquoten. \cite[S. 1]{Cortez2008UsingDM}
Daher wurde mit dieser Arbeit ein realer Datensatz erhoben.
Hierfür wurden Schulleisten und schulbezogene Faktoren vom Berichtswesen gesammelt und demografische und soziale Faktoren durch Befragungen ermittelt. \cite[S. 1]{Cortez2008UsingDM}
Die schulbezogenen Faktoren beziehen sich dabei auf die Leistungen in den Schulfächern Mathematik und Portugiesisch, da Inhalte dieser Fächer übergreifend in anderen Fächern zum Einsatz kommen. \cite[S. 2]{Cortez2008UsingDM}
Zielgruppe der Untersuchung waren Schüler der dreijährigen zweiten Bildungsphase in Portugal, welche auf der ersten neunjährigen Phase aufbaut. \cite[S. 2]{Cortez2008UsingDM}
Mittels der Berichte und Umfragen konnten schulbezogene-, demografische-, und soziale Faktoren von X Mathematik-Schülern und Y Portugiesisch-Schülern der beiden Schulen erhoben werden.
Alle erhobenen Faktoren, welche im Rahmen der Analyse auch als Variablen oder Merkmale bezeichnet werden, können der nachfolgenden Liste entnommen werden. 

%\begin{table}[!ht]
    \centering
    \begin{tabular}{lll}
    \hline
        \textbf{Beschreibung} & \textbf{Datentyp} & \textbf{Ausprägungen} \\ \hline
        Schulbezeichnung & Binär & GP - Gabriel Pereira; MS - Mousinho da Silveira \\ 
        Geschlecht & Binär & F - female; M - male \\ 
        Alter & Numerisch & 15 to 22 \\ 
        Wohngegend & Binär & U - urban; R - rural \\ 
        Familiengröße & Binär & LE3 - less or equal to 3; GT3 - greater than 3 \\ 
        Zusammenleben der Eltern & Binär & T - living together; A - apart \\ 
        Mutters Bildungsgrad & Numerisch & 0 - none; 1 - primary education (4th grade); 2 – 5th to 9th grade; 3 – secondary education; 4 – higher education \\ 
        Vaters Bildungsgrad & Numerisch & 0 - none; 1 - primary education (4th grade); 2 – 5th to 9th grade; 3 – secondary education; 4 – higher education \\ 
        Mutters Arbeitsbereich & Nominal & teacher; health care related; civil services (administrative or police); at home; other \\ 
        Vaters Arbeitsbereich & Nominal & teacher; health care related; civil services (administrative or police); at home; other \\ 
        Grund der Schulentscheidung & Nominal & close to home; school reputation; course preference; other \\ 
        Erziehungsberechtigter & Nominal & mother; father; other \\ 
        Pendelzeit & Numerisch & 1 - <15 min.; 2 - 15 to 30 min.; 3 - 30 min. to 1 hour; 4 - >1 hour \\ 
        Wöchentliche Lernzeit & Numerisch & 1 - <2 hours; 2 - 2 to 5 hours; 3 - 5 to 10 hours; 4 - >10 hours \\ 
        Anzahl durchgefallener Kurse & Numerisch & 1 to 3; 4 \\ 
        externe Lernunterstützung & Binär & yes; no \\ 
        familiäre Lernunterstützung & Binär & yes; no \\ 
        Extra bezahlter Unterricht im befragungsfach & Binär & yes; no \\ 
        außerschulische Aktivitäten & Binär & yes; no \\ 
        Besuch der Vorschule & Binär & yes; no \\ 
        Absicht zur Weiterbildung & Binär & yes; no \\
        häuslicher Internetzugang & Binär & yes; no \\
        Partnerliche Beziehung & Binär & yes; no \\
        Qualität der Familienbeziehungen & Numerisch & from 1 - very bad to 5 - excellent\\
        außerschulische Freizeit & Numerisch & from 1 - very low to 5 - very high \\
        soziale Aktivitäten & Numerisch & from 1 - very low to 5 - very high \\
        Alkoholkonsum unter der Woche & Numerisch & from 1 - very low to 5 - very high \\
        Alkoholkonsum am Wochenende & Numerisch & from 1 - very low to 5 - very high \\
        aktueller Gesundheitszustand & Numerisch & from 1 - very bad to 5 - very good\\
        Anzahl der Fehltage & Numerisch & from 0 to 93 \\
        erste Vorabnote & Numerisch & from 0 to 20 \\
        zweite Vorabnote & Numerisch & from 0 to 20 \\
        finale Note & Numerisch & from 0 to 20 \\
    \hline
    \end{tabular}
\end{table}


\begin{table}[htb]
    \centering
    \begin{tabular}{|l|l|l|}
    \hline
        Numerisch & \makecell[l]{Alter; Mutters Bildungsgrad; Vaters Bildungsgrad; Pendelzeit; \\Wöchentliche Lernzeit; Anzahl durchgefallener Kurse; \\Qualität der Familienbeziehungen; außerschulische Freizeit; \\soziale Aktivitäten; Alkoholkonsum unter der Woche; \\Alkoholkonsum am Wochenende; \\aktueller Gesundheitszustand; Anzahl der Fehltage; \\\textbf{erste Vorabnote; zweite Vorabnote; finale Note}} \\ \hline
        Binär & \makecell[l]{Schulbezeichnung; Geschlecht; Wohngegend; Familiengröße; \\Zusammenleben der Eltern; externe Lernunterstützung; \\familiäre Lernunterstützung; \\Extra bezahlter Unterricht im Befragungsfach; \\außerschulische Aktivitäten; Besuch der Vorschule; \\Absicht zur Weiterbildung; häuslicher Internetzugang; \\Partnerliche Beziehung} \\ \hline
        Nominal & \makecell[l]{Mutters Arbeitsbereich; Vaters Arbeitsbereich; \\Grund der Schulentscheidung; Erziehungsberechtigter} \\ \hline
    \end{tabular}
\end{table}

Detaillierte Informationen zu dessen Erläuterung und ihren Ausprägungen der \textbf{Tabelle X} im Anhang entnommen werden.

\section[Forschungsfragen]{Forschungsfragen}

\begin{itemize}
    \item Intro zur Absicht dieser Sektion
    \item Erläuterung der Absicht mit Bezug auf die Zielgruppe dieser Arbeit
    \item Einleitung zu den bereits erkannten Variablen ?
    \item Ableitung von interessanten Forsch
\end{itemize}
\chapter[Grundlagen]{Grundlagen}
\label{chap:literature}

\section[how to Visualisierungen ?]{how to Visualisierungen ?}
\section[how to analyse Visualisierungen ?]{how to analyse Visualisierungen ?}

\chapter[Untersuchung der X]{Untersuchung der X}
\chapter[Fazit]{Fazit}


\begin{itemize}
    \item Lernleistung lediglich durch Noten bemessen
    \item alter Datensatz
    \item Komplexität ist ok, da durch das digitale format Visualisierungen in den angezeigten Daten gefiltert werden können. (zoom, kategorie auswahl)
\end{itemize}

%Literaturverzeichnis einfügen
\renewcommand*{\UrlFont}{\rmfamily}
\printbibliography
\chapter[Anhang]{Anhang}

\begin{table}[!ht]
    \centering
    \begin{tabular}{lll}
    \hline
        \textbf{Beschreibung} & \textbf{Datentyp} & \textbf{Ausprägungen} \\ \hline
        Schulbezeichnung & Binär & GP - Gabriel Pereira; MS - Mousinho da Silveira \\ 
        Geschlecht & Binär & F - female; M - male \\ 
        Alter & Numerisch & 15 to 22 \\ 
        Wohngegend & Binär & U - urban; R - rural \\ 
        Familiengröße & Binär & LE3 - less or equal to 3; GT3 - greater than 3 \\ 
        Zusammenleben der Eltern & Binär & T - living together; A - apart \\ 
        Mutters Bildungsgrad & Numerisch & 0 - none; 1 - primary education (4th grade); 2 – 5th to 9th grade; 3 – secondary education; 4 – higher education \\ 
        Vaters Bildungsgrad & Numerisch & 0 - none; 1 - primary education (4th grade); 2 – 5th to 9th grade; 3 – secondary education; 4 – higher education \\ 
        Mutters Arbeitsbereich & Nominal & teacher; health care related; civil services (administrative or police); at home; other \\ 
        Vaters Arbeitsbereich & Nominal & teacher; health care related; civil services (administrative or police); at home; other \\ 
        Grund der Schulentscheidung & Nominal & close to home; school reputation; course preference; other \\ 
        Erziehungsberechtigter & Nominal & mother; father; other \\ 
        Pendelzeit & Numerisch & 1 - <15 min.; 2 - 15 to 30 min.; 3 - 30 min. to 1 hour; 4 - >1 hour \\ 
        Wöchentliche Lernzeit & Numerisch & 1 - <2 hours; 2 - 2 to 5 hours; 3 - 5 to 10 hours; 4 - >10 hours \\ 
        Anzahl durchgefallener Kurse & Numerisch & 1 to 3; 4 \\ 
        externe Lernunterstützung & Binär & yes; no \\ 
        familiäre Lernunterstützung & Binär & yes; no \\ 
        Extra bezahlter Unterricht im befragungsfach & Binär & yes; no \\ 
        außerschulische Aktivitäten & Binär & yes; no \\ 
        Besuch der Vorschule & Binär & yes; no \\ 
        Absicht zur Weiterbildung & Binär & yes; no \\
        häuslicher Internetzugang & Binär & yes; no \\
        Partnerliche Beziehung & Binär & yes; no \\
        Qualität der Familienbeziehungen & Numerisch & from 1 - very bad to 5 - excellent\\
        außerschulische Freizeit & Numerisch & from 1 - very low to 5 - very high \\
        soziale Aktivitäten & Numerisch & from 1 - very low to 5 - very high \\
        Alkoholkonsum unter der Woche & Numerisch & from 1 - very low to 5 - very high \\
        Alkoholkonsum am Wochenende & Numerisch & from 1 - very low to 5 - very high \\
        aktueller Gesundheitszustand & Numerisch & from 1 - very bad to 5 - very good\\
        Anzahl der Fehltage & Numerisch & from 0 to 93 \\
        erste Vorabnote & Numerisch & from 0 to 20 \\
        zweite Vorabnote & Numerisch & from 0 to 20 \\
        finale Note & Numerisch & from 0 to 20 \\
    \hline
    \end{tabular}
\end{table}


\include{chapter/99_eidesstattliche-erklaerung.tex}
\end{document}
