\chapter[Grundlagen]{Grundlagen}
\label{chap:literature}

Abgeleitet von der zuvor gestellten Forschungsfragen werden in diesem Kapitel notwendige wissenschaftliche Grundlagenkonzepte erläutert.
Diese Grundlagenkonzepte entstammen den Forschungsfeldern Data Mining im Bildungsbereich und Business Analytics.

\section[Data Mining im Bildungsbereich]{Data Mining im Bildungsbereich}

Data Mining im Bildungsbereich bedeutet, den analytischen Prozess zu beleuchten, welcher Forschenden ermöglicht aus großen Mengen von Rohdaten wertvolles Wissen über bildungspolitische Maßnahmen und Praktiken zu erlangen (\cite[S. 1]{Gamazo.2020}).
Mit der Erforschung dieses Prozesses werden die vier Ziele der Lernverhaltensvorhersage von Schülern, der Erweiterung von Methoden der Psychometrie, das Schaffen von Bildungstemplates und Tools sowie der Ergründung der Wirkung von Lernunterstützung verfolgt (\cite[S. 2]{P.Bachhal.2021}).
Das Interesse an diesem Gebiet wird durch die Herausforderung des nicht linearen Einflusses auf die Bildungsleistung durch vielerlei Faktoren geprägt (\cite[S. 1]{Khan.2021}).
Vorhersage, Beziehungsmining und Strukturaufdeckung bilden die bekanntesten Methoden des Data Minings im Bildungsbereich (\cite[S. 3]{Gamazo.2020}).
Anhand von Beziehungsmining werden die stärksten Beziehungen zwischen Merkmalen eines Datensatzes ergründet, ohne vorherige Festlegung von Kriterien oder Prädiktorvariablen (\cite[S. 3]{Gamazo.2020}).

\section[Business Analytics]{Business Analytics}

In der Praxis werden Erkenntnisse der Forschung zu Business Analytics dafür eingesetzt, um mittels Organisationsdaten strategische und operationelle Entscheidungen zu fundieren (\cite[S. 1]{Qin.2020}).
Diagramme erfüllen dabei den Zweck, ausgerichtet auf Strukturen der visuellen Wahrnehmung, die Identifikation von Mustern zu vereinfachen (\cite[S. 287]{Baars2021}).
Informationswahrnehmung erfolgt durch die Assoziation, Differenzierung und Ordnung von Objekten sowie durch deren Quantität und Größe (\cite[S. 287]{Baars2021}).
Die Gestaltung der Informationswahrnehmung wird zusätzlich unterstützt je konsistenter sich die Darstellung mit dem Vorwissen des Betrachters deckt (\cite[S. 288]{Baars2021}).
Wichtige Schritte im Prozess der Gestaltung umfassen die Manipulation (z. B. Filterung, Aggregation) und das Kartieren (z. B. Zuordnung zu Formen) von Daten (\cite[S. 2]{Qin.2020}).
Die Sprache von Visualisierungen besteht somit aus transformierten Daten, in Art, Größe und Farbe unterschiedliche Markierungen und deren Abbildungsverhältnis (\cite[S. 3]{Qin.2020}).