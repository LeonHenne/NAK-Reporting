\chapter[Untersuchung der Forschungsfrage]{Untersuchung der Forschungsfrage}

Dieses Kapitel dient der Untersuchung der Forschungsfragen. 
In Abhängigkeit des untersuchten Handlungsbereichs werden hierfür notwendige Datentransformationen erläutert. 
Anschließend gilt es aufgrund der Vielzahl an Attributen, die für die Untersuchung relevanten Merkmale zu identifizieren. 
Final wird die Beantwortung der Forschungsfrage, durch die Vorstellung und Analyse der entwickelten Visualisierungen vorgenommen.
Je nach Präsentationsart empfiehlt sich die Darstellung der Visualisierungen anhand exportierter HTML Webseiten oder des im Anhang verfügbaren Jupyter Notebooks.
Dies ermöglicht eine Interaktion mit den Visualisierungen durch Funktionen wie dem Hineinzoomen und der Filterung von Kategorien.
Im Rahmen dieser Arbeit werden entsprechend dem PDF-Format lediglich statische Abbildungen hinzugefügt.

\section{Individuelle und strukturelle Faktoren}

% Erläuterung der Datenstruktur: merged, left join, individual (mat\_df, por\_df)
Zur Untersuchung dieser Faktoren eignet sich besonders ein fachübergreifende Perspektive. 
Dies resultiert daraus, dass eine unterschiedliche Auswirkung auf die Mathematik- oder Portugiesischlehre nur schwierig kausal zu erklären ist.
Die Migrierung beider Datenstände wird anhand der Attribute vorgenommen, welche im, unter (\cite[]{student_performance}) verfügbaren, R-Skript benannt werden.
Insgesamt erhält der Datensatz damit die 382 Einträge der Schüler, welche Teil beider Befragungen sind. 
Sofern unter den ausgewählten Attributen nominale Merkmale vorliegen, welche nicht bei der Datenmigrierung berücksichtigt wurden, kann es hier zu widersprüchlichen Angaben kommen.
Diese widersprüchlichen Dateneinträge können nicht ohne weiteres Wissen aufgeklärt, oder ein Mittelmaß bestimmt werden und sind daher zusätzlich aus der Betrachtung zu nehmen.

% \item Auswahl der Variablen in Abhängigkeit zur Frage:
Die betrachteten Faktoren lassen sich nach individuellen Verhaltensweisen und strukturellen Gegebenheiten aufteilen.
Individuelle Merkmale werden nach der Möglichkeit zur Einflussnahme durch den Schüler gefiltert.
Weiterhin tritt die Absicht zur Weiterbildung zwar in diesem Kontext als möglicher Prädiktor des Lernerfolgs auf, die alleinige Absicht selbst nimmt jedoch hierauf keinen unmittelbaren Einfluss.
Damit werden die Attribute Alkoholkonsum an Arbeitstagen und Alkoholkonsum am Wochenende in die Untersuchung einfließen.
Strukturelle Gegebenheiten werden zu dieser Fragestellung durch den Grund der Schulentscheidung, die Pendelzeit und den häuslichen Internetzugang beschrieben.
Aus diesen Merkmalen kann besonders der Pendelzeit und dem häuslichen Internetzugang ein potenzieller kausaler Zusammenhang unterstellt werden, anhand des täglichen zusätzlichen Reiseaufwands und dem Zugang zu Onlinewissen.
Daher werden diese beiden Attribute für die Untersuchung ausgewählt.

\clearpage
\begin{figure}[htb]
    \centering
    \includegraphics[width=1.0\textwidth]{src/visuals/image/stacked_bar.png}
    \caption{Einfluss von Alkoholkonsum auf die Lernleistung}
    \label{fig:stacked_bar}
\end{figure}

% Vorstellung der Visualisierung
Folgend wird der Beantwortung der Forschungsfrage zum Einfluss individueller und struktureller Faktoren auf die Lernleistung eines Schülers nachgegangen. 
Hierzu dient das in Abbildung \ref{fig:stacked_bar} dargestellte gestapelte Säulendiagramm.
Anhand dessen wird als erstes erforscht, wie individuelle Verhaltensweisen auf die Lernleistung wirken anhand des wöchentlichen Alkoholkonsums.
Mit dieser Diagrammart wird die Aggregation von Daten notwendig, womit jedoch auch eine Vereinfachung der Verständlichkeit einhergeht.
Dem Teil der Zielgruppe ohne statistischen Hintergrund ermöglicht dies besonders zu Beginn einer Präsentation einen leichteren Zugang zu der dargelegten Thematik.
Die Grafik ordnet auf der X-Achse die Schüler anhand ihres Medians der finalen Noten beider Fächer in Gruppen ein.
Auf der Y-Achse wird der Median des von den Schülern angegebenen Alkoholkonsums auf der Skala von null bis zehn abgetragen.
Der Wertebereich ergibt sich aus der Addition der Maximalwerte beider Kategorien.
In der Grafik werden die Kategorien des Alkoholkonsums an Arbeitstagen und an Wochenenden berücksichtigt.
Farblich sind diese Kategorien zu den Farben von typischen Alkoholprodukten wie Wein und Bier gekennzeichnet, um die Zuordnung zum Thema zu unterstreichen.
Zusätzlich können die Farben zu Warnstufen einer Ampel assoziiert werden, bei dem ein Alkoholkonsum an Werktagen häufig kritischer eingeschätzt wird, als an Wochenendtagen.
Die Höhe der Säule einer Schülergruppe ergibt sich folglich aus der Summe der Mediane beider Kategorien, um so den gesamten Alkoholkonsum einer Gruppe korrekt darzustellen.

% Analyse der Visualisierung
Der Abbildung \ref*{fig:stacked_bar} kann entnommen werden, dass die Schülergruppen nahezu alle einen sehr geringen Alkoholkonsum an Werktagen aufweisen, und sich hauptsächlich durch ihren Konsum am Wochenende unterscheiden.
Eine Ausnahme bildet dabei die Gruppe, welche im Median beider Fächer die geringste Punktzahl aufweist. 
Hier liegt bereits der Konsum an Werktagen als einziges leicht über allen anderen Gruppen.
Nimmt man die zweite Kategorie mit in die Betrachtung auf, zeigt dieselbe Gruppe erneut einen stärkeren Alkoholkonsum auf.
Insgesamt liegt die Gruppe mit der geringsten Medianpunktzahl mit 1.5 Einschätzungspunkten über allen anderen Gruppen, was einen negativen Effekt des Alkoholkonsums auf die Lernleistung vermuten lässt.
Unterstützt wird diese Vermutung durch die Untersuchung der weiteren Gruppen. 
Der Verlauf des Alkoholkonsums zeigt hier, dass der Konsum am Wochenende ab einer Mediannote von 14 und 18 Punkten stufenweise absinkt. 
Eine Ausnahme hierzu stellt die Gruppe im Bereich von vier bis sechs Mediannotenpunkten dar, dessen Konsum gleich dem der Gruppe mit der höchsten Mediannotenpunktzahl ist.

\begin{figure}[htb]
    \centering
    \includegraphics[width=1.0\textwidth]{src/visuals/image/violin.png}
    \caption{Zusammenhang struktureller Faktoren und Lernleistungen}
    \label{fig:violin}
\end{figure}

% Vorstellung der Visualisierung
Strukturelle Einflüsse auf die Lernleistung werden in dieser Arbeit durch Betrachtung der Pendelzeit und des heimischen Internetzugangs untersucht.
Für die Untersuchung dient der in Abbildung \ref{fig:violin} dargelegte Violinen-Plot.
Dargestellt wird mit dieser Diagrammart die Datenverteilung der im Median erreichten Notenpunktzahl in Portugiesisch und Mathematik, im Bezug auf die gewählten Variablen.
Hierbei legt die X-Achse die Pendelzeit anhand ihrer vier möglichen Werte (\textit{<15 min, 15-30 min, 30-60 min, >60 min}) und die Y-Achse die im Median erreichte Notenpunktzahl beider Fächer dar.
Eine \textit{Violine} wird dabei in der Verteilung linksseitig durch die der Gruppe des vorhandenen Internetzugangs und rechtsseitig durch die Gruppe des fehlenden Internetzugangs bestimmt.
Zu jeder Verteilung sind dessen Median als gestrichelte Linie und Ausreißer als Punkte erkennbar.
Die Farbe einer Verteilung orientiert sich hier anhand der Kategorie und folgt dem allgemeinen Farbverständnis von Grün als Bestätigung und Rot als Ablehnung des Internetzugangs.

% Analyse der Visualisierung
Mittels der Untersuchung der Abbildung \ref{fig:violin} und dessen Verlauf über alle Pendelzeiten wird zunächst deutlich, dass die gezeigten Mittelwerte mit steigender Pendelzeit tendenziell abnehmen. 
Wird hinzu die Verteilung der Lernleistung betrachtet, ist dieser Trend aufgrund der hohen Quartilsabstände teilweise zu relativieren.
Einzige Ausnahme dieses Abnahmetrends zeigt die Schülergruppe, welche eine Pendelzeit von mehr als 60 Minuten besitzt sowie über heimischen Internetzugang verfügt.
Besonders deutlich in dieser Pendelzeitgruppe wird zudem der im Verhältnis hohe Abstand von 4.25 Notenpunkten zwischen den Mittelwerten in Abhängigkeit des Internetzugangs.
Betont wird dies zugleich durch die deutlich geringeren Quartilsabstände der Notenverteilung in dieser Pendelzeitgruppe.
Da ohne weiteres Wissen lediglich Vermutungen über die kausalen Gründe dieses Phänomens aufgestellt werden können, gilt es dieses in weiteren Arbeiten detaillierter zu Erforschen.

% TODO: Evtl. Forschungsfrage beantworten, ansonsten im Fazit.

\section{Soziales Umfeld}

% \item Erläuterung der Datenstruktur: merged, left join, individual (mat\_df, por\_df)
Der Einfluss des sozialen Umfelds eines Schülers auf dessen Lernleistung wird gleich der vorherigen Sektion aus fachübergreifender Perspektive betrachtet.
Dadurch kommen für die folgende Untersuchung die gleichen Datentransformationen zur Trage.

% \item Auswahl der Variablen in Abhängigkeit zur Frage:
In der Auswahl von Attributen wird sich zum einen auf das familiäre Umfeld beschränkt. 
Dabei werden konkret die Merkmale Familiengröße und Qualität der Familienbeziehungen untersucht und die Information zum Zusammenleben der Eltern, deren Bildungsgrad und Arbeitsbereich und zum Sorgerecht exkludiert.
Diese Entscheidung begründet sich in der Annahme, dass sich die Effekte der ausgeschlossenen Variablen in der Qualität der Familienbeziehungen konsolidiert widerspiegeln.
Zum anderen werden die Variablen Partnerstatus und Intensität sozialer Aktivitäten der Analyse hinzugefügt, um das soziale Verhältnis mit Gleichaltrigen zu Beleuchten.

\begin{figure}[htb]
    \centering
    \includegraphics[width=1.0\textwidth]{src/visuals/image/bar.png}
    \caption{Einfluss der Familiengröße und dessen Beziehungsqualität auf die Lernleistung}
    \label{fig:bar}
\end{figure}

% Vorstellung der Visualisierung
Abbildung \ref{fig:bar} zeigt ein Säulendiagramm, welches die Beziehung zwischen Familiengröße dessen Beziehungsqualität und der Lernleistung thematisiert.
Daran wird folglich die Forschungsfrage nach dem Einfluss des sozialen Umfelds auf die Lernleistung genauer beleuchtet.
Lernleistung wird als Median der erreichten Notenpunktzahl beider Fächer auf der Y-Achse im vollen möglichen Bereich von null bis 20 dargestellt.
Dabei wird der allgemeine Konsens angenommen, das potenziell abhängige Variablen auf dieser Y-Achse dargelegt werden, um der Gewohnheit der Zielgruppe zu entsprechen.
Die X-Achse erfüllt zugleich den Zweck, den Median der Qualität der Familienbeziehungen abzubilden.
Durch die Aggregation der Daten aus den zwei Befragungen entstehen hieraus auch ungerade Werte und damit einzelne Säulen der Beziehungsqualität in lediglich einer Familiengrößenkategorie.
Kategorien der Familiengröße werden zu jeder Beziehungsqualitätsgruppe als Säulen, farblich möglichst kontrastreich nebeneinander platziert, was dem Betrachter den Vergleich dieser Werte vereinfacht. 

% Analyse der Visualisierung
Die Analyse der Abbildung \ref{fig:bar} zeigt über alle Qualitäten der Familienbeziehungen nahezu die gleichen Werte der im Median erreichten Notenpunkte.
Aus der Betrachtung dieses Datensatzes kann somit kein Einfluss der Güte der Familienbeziehung auf die Lernleistung in Form von besseren Schulnoten vermutet werden.
Hingegen ist der Grafik das Phänomen zu entnehmen, dass zumindest unter besseren Familienbeziehungen und einer größeren Anzahl von Familienmitgliedern auch eine geringe höhere Lernleistung auftritt.
Zweck weiterer Arbeiten könnte es damit sein, dieses wie auch bisherige Phänomene anhand neuer Analysen tiefgehender zu beleuchten.

\begin{figure}[htb]
    \centering
    \includegraphics[width=1.0\textwidth]{src/visuals/image/box.png}
    \caption{Lernleistungseinfluss des Partnerstatus und sozialer Aktivitäten}
    \label{fig:box}
\end{figure}

% Vorstellung der Visualisierung
Die Untersuchung des freundschafts- und partnereinflusses auf die Lernleistung wird anhand der Abbildung \ref{fig:box} durchgeführt.
Darin wird zu jedem Wert der unabhängigen Variablen die Verteilung der abhängigen Variable als Boxplot abgebildet.
Erläuterungen zu der Darstellungsform des Boxplots wurden hierzu in \autoref{chap:literature} abgehandelt.
Entsprechend der häufigen Darstellung werden die unabhängigen Variablen auf der X-Achse dargestellt.
Dies erfolgt zum einen durch das Aufzeigen aller möglichen Werte (eins bis fünf) der Intensität sozialer Aktivitäten.
Zum anderen werden jedem Intensitätsgrad zwei Boxplots entsprechend der möglichen Partnerstatus zugewiesen.
Ein Partnerstatus wird dabei farblich kodiert, sodass das Vorhandensein eines Partners entsprechend der damit einhergehenden Assoziation mit Liebe, zu einem tiefroten Boxplot führt.
Das Fehlen eines Partners wird durch das soziale Grundbedürfnis eines Menschen als tendenziell einsamer und weniger aufregend eingeschätzt und damit durch die graue Farbe ausgedrückt.

% Analyse der Visualisierung
Folgend wird die Korrelation der abhängigen Variable ({Median der erreichten Note}) und der unabhängigen Variablen ({Intensität sozialer Aktivitäten}, {Partnerstatus}) untersucht.
Der Erkenntnisgewinn korrelierender Effekte gilt es in weiteren Arbeiten auf kausalen Zusammenhang zu prüfen.
Ein solcher korrelierender Effekt kann die bei der Betrachtung der Intensität sozialer Aktivitäten vermutet werden. 
Hierbei zeigt sich, bis auf bei der Gruppe mit einer sehr geringen Aktivität, ein sehr leicht abnehmender Trend der im Median erreichten Note, unter steigender Intensität an Sozialaktivitäten.
Weiterhin auffällig ist dazu die Inspektion des Partnerstatus bei sehr geringer sozialer Aktivität.
In dieser Gruppe von Schülern zeigt sich im Median eine Verringerung der Notenpunktzahl um vier Punkte unter denen, welche eine partnerliche Beziehung führen.
Kausale Vermutungen, wie das eine partnerliche Beziehung unter zugleich geringer anderweitiger sozialer Aktivitäten zu mehr Streit und damit geringerer Lernleistung führt, bedarf einer Validierung durch Analysen weiterer Arbeiten.

% TODO: Evtl. Forschungsfrage beantworten, ansonsten im Fazit.

\section{Individuelle Leistungsbereitschaft}

% \item Erläuterung der Datenstruktur: merged, left join, individual (mat\_df, por\_df)
In der Untersuchung des Einflusses individueller Leistungsbereitschaft auf die Lernleistung werden die ursprünglichen fachspezifischen Datensätze verwendet.
Grund dafür ist der fehlende kausale Zusammenhang zwischen der Leistungsbereitschaft in einem Fach und der erzielten Lernleistung im anderen Fach.
Zu Berücksichtigen sind damit die unterschiedliche Datensatzgröße mit 395 (Mathematik-Schüler) und 649 (Portugiesisch-Schüler) Einträgen.
Entsprechend werden in dieser Sektion alle entwickelten Visualisierungen jeweils mit beiden Datengrundlagen dargestellt.

% \item Auswahl der Variablen in Abhängigkeit zur Frage:
    % \subitem studytime, failures, activities, schoolsup, famsup, paid, absences
Von den, der Forschungsfrage zugeordneten Attributen, zeigen Lernzeit, Anzahl bereits durchgefallener Kurse, externe Lernunterstützung, familiäre Lernunterstützung, bezahlter Extraunterricht und Fehltage eine besondere Relevanz für die Lernleistung. 
In dieser Ergründung der Einflussfaktoren wird die AG-Teilnahme exkludiert, da hiermit zwar Zusatzarbeiten, jedoch abseits vom fachlichen Thema, geleistet werden.
Zusätzlich wird die Anzahl bereits durchgefallener Kurse nicht beachtet, weil hierbei kein kausaler Zusammenhang zur aktuellen Lernleistung besteht.
\clearpage

\begin{figure}[htb]
    \centering
    \includegraphics[width=1.0\textwidth]{src/visuals/image/balk.png}
    \caption{Einfluss von Lernunterstützungen zu den Lernleistungen von Schülern}
    \label{fig:balk}
\end{figure}

% Vorstellung der Visualisierung
Mit Abbildung \ref{fig:balk} wird dem Einfluss der individuellen Leistungsbereitschaft auf die Lernleistung nachgegangen.
Wie zu Beginn dieser Sektion erläutert, wird keine fachübergreifende Perspektive eingenommen, weshalb folglich zwei Grafiken dargestellt sind.
Mittels der Grafiken werden visuell die Abhängigkeitsbeziehung der Lernleistung von familiär, schulisch oder bezahlt erhaltenen Lernunterstützungen ergründet.
Eine Grafik enthält ein Boxplot für jeden Wert der abhängigen Variable, welcher auf der Y-Achse die zugehörige Verteilung der erreichten Note anzeigt.
Die visuelle Skala der Y-Achse entspricht dabei der möglichen Punkteskala einer Note.
Durch die zweierlei gleiche Skala der Y-Achse kann dem Betrachter der Zielgruppe zugemutet werden zu erkennen, dass hier lediglich eine Y-Achsenbeschriftung für beide Grafiken gilt.
Damit kann der Abbildung Text entnommen, und so diese übersichtlicher gestaltet werden.
Auf der X-Achse wird die Summe der von einem Schüler erhaltenen Lernunterstützung, entsprechend in der Skala von null bis drei abgebildet.
Die Boxplots beider Grafiken sollen vom Betrachter durch ihre Farbgebung möglichst leicht den einzelnen Schulfächern zugeordnet werden können.
Unterstützt wird dies damit, dass die Grafik der Befragung im Fach Portugiesisch farblich mit dem Rot der portugiesischen Flagge gekennzeichnet ist (\cite[]{Wikipedia.2024}).
Dazu wurde die Grafik der Befragung im Fach Mathematik möglichst kontrastreich in Blau markiert.

% Analyse der Visualisierung
Die Analyse der Abbildung \ref{fig:balk} zeigt zwei hervorzuhebende Besonderheiten auf.
Anhand der Beobachtung des Mittelwerts kann im Fall beider Lernleistungen vermutet werden, dass allgemein mit höherem Erhalt von Lernunterstützung, die Lernleistung wider erwartend abnimmt oder maximal gleich bleibt.
Weiterhin wird durch die Quartilsabstände vermutbar, dass Lernunterstützungen sogar fachübergreifend die Streuung der Lernleistung deutlich verringern können.

\begin{figure}[htb]
    \centering
    \includegraphics[width=1.0\textwidth]{src/visuals/image/nextbar.png}
    \caption{Lernleistung anhand der Lernzeit und Anzahl der Schulfehltage}
    \label{fig:nextbar}
\end{figure}
% TODO: FASS DICH KURZ !!! - bisher ~2700 Wörter !!! Grundlagen und Fazit fehlen auch noch

% Vorstellung der Visualisierung
Ein weiterer Aspekt der dritten Forschungsfrage wird durch Abbildung \ref{fig:nextbar} analysiert.
Beide Grafiken zeigen die Beziehung auf der Lernzeit und der Menge der Fehltage zu der erreichten Note.
Die X-Achse einer Grafik drückt dabei zum einen die vier Werte der wöchentlichen Lernzeit aus.
Über mehrere Säulen wird die Menge der Fehltage in vier Bereiche kategorisiert. 
Die Farben der Säulen sind im Verlauf von Grün zu Rot dargestellt, um die Gefährdung durch Wissensbenachteiligung zu signalisieren.

% Analyse der Visualisierung
Über beide Grafiken hinweg lässt sich erkennen, dass in mehreren Kategorien eine Erhöhung der Lernzeit zu einer gesteigerten Lernleistung führt.
Die Lernleistung, widergespiegelt in der Note, folgt undeutlicher dem Trend, dass eine zunehmende Abwesenheit geringere Noten hervorruft.
Dieser Effekt tritt nur im Mathematikbereich bis zu einer Lernzeit von fünf Stunden auf.
Hingegen klarer erkenntlich ist, dass unabhängig der betrachteten Variablen die Noten in Mathematik im Median geringer sind, als in Portugiesisch.

% TODO: Evtl. Forschungsfrage beantworten, ansonsten im Fazit.