\chapter[Untersuchung der Forschungsfrage]{Untersuchung der Forschungsfrage}

Dieses Kapitel dient der Untersuchung der Forschungsfragen. 
In Abhängigkeit des untersuchten Handlungsbereichs werden die hierfür notwendigen Datentransformationen erläutert. 
Anschließend gilt es aufgrund der Vielzahl an Attributen, die für die Untersuchung relevanten Merkmale zu identifizieren. 
Final wird die Beantwortung der Forschungsfrage, durch die Vorstellung und Analyse der entwickelten Visualisierungen vorgenommen.

\section{Individuelle und strukturelle Faktoren}

% Erläuterung der Datenstruktur: merged, left join, individual (mat\_df, por\_df)
Zur Untersuchung dieser Faktoren eignet sich besonders ein fachübergreifende Perspektive. 
Dies resultiert daraus, dass eine unterschiedliche Auswirkung auf die Mathematik- oder Portugiesischlehre nur schwierig kausal zu erklären ist.
Die Migrierung beider Datenstände wird anhand der Attribute vorgenommen, welche im, unter (\cite[]{student_performance}) verfügbaren, R-Skript benannt werden.
Insgesamt erhält der Datensatz damit die 382 Einträge der Schüler, welche Teil beider Datensätze sind. 
Sofern unter den ausgewählten Attributen nominale Merkmale vorliegen, welche nicht bei der Datenmigrierung berücksichtigt wurden, kann es hier zu widersprüchlichen Angaben kommen.
Diese widersprüchlichen Dateneinträge können nicht ohne weiteres Wissen aufgeklärt, oder ein Mittelmaß daraus bestimmt werden und sind daher zusätzlich aus der Betrachtung zu nehmen.

% \item Auswahl der Variablen in Abhängigkeit zur Frage:
Die betrachteten Faktoren lassen sich nach individuellen Verhaltensweisen und der strukturellen Gegebenheiten aufteilen.
Individuelle Merkmale werden nach der Möglichkeit zur Einflussnahme durch den Schüler gefiltert.
Weiterhin tritt die Absicht zur Weiterbildung zwar in diesem Kontext als möglicher Prädiktor des Lernerfolgs auf, die alleinige Absicht selbst nimmt jedoch hierauf keinen unmittelbaren Einfluss.
Damit werden die Attribute \textit{Alkoholkonsum an Arbeitstagen, Alkoholkonsum am Wochenende} in die Untersuchung einfließen.
Strukturelle Gegebenheiten werden im Kontext dieser Arbeit durch den Grund der Schulentscheidung, die Pendelzeit und den häuslichen Internetzugang beschrieben.
Aus diesen Merkmalen kann besonders der \textit{Pendelzeit} und dem \textit{häuslichen Internetzugang} ein potenzieller kausaler Zusammenhang unterstellt werden, anhand des täglichen zusätzlichen Reiseaufwands und dem Zugang zu Onlinewissen.
Daher werden diese beiden Attribute für die Untersuchung ausgewählt.

\clearpage
\begin{figure}[htb]
    \centering
    \includegraphics[width=1.0\textwidth]{src/visuals/image/stacked_bar.png}
    \caption{Einfluss von Alkoholkonsum auf die Lernleistung}
    \label{fig:stacked_bar}
\end{figure}

% Vorstellung der Visualisierung
Die Untersuchung des Einflusses individueller Verhaltensweisen auf die Lernleistung wird u. a. anhand des wöchentlichen Alkoholkonsums vorgenommen.
Hierzu dient das in Abbildung \ref{fig:stacked_bar} dargestellte gestapelte Säulendiagramm.
Mit dieser Diagrammart wird die Aggregation von Daten notwendig, womit jedoch auch eine Vereinfachung der Verständlichkeit einhergeht.
Dem Teil der Zielgruppe ohne statistischen Hintergrund ermöglicht dies besonders zu Beginn einer Präsentation einen leichteren Zugang zu der dargelegten Thematik.
Die Grafik ordnet auf der X-Achse die Schüler anhand ihres Medians der finalen Noten beider Fächer in Gruppen von je zwei Punkten ein.
Auf der Y-Achse wird der Median des von den Schülern angegebenen Alkoholkonsums auf der Skala Null bis Zehn abgetragen.
Der Wertebereich ergibt sich aus der Addition der Maximalwerte beider Kategorien.
In der Grafik werden die Kategorien des Alkoholkonsums an Arbeitstagen und an Wochenenden berücksichtigt.
Farblich sind diese Kategorien zu den typischen Farben von Alkoholprodukten wie Wein und Bier gekennzeichnet, um die Zuordnung zum Thema zu unterstreichen.
Zusätzlich können die Farben den Warnstufen einer Ampel zugeordnet werden, bei dem ein Alkoholkonsum an Werktagen häufig kritischer eingeschätzt wird, als an Wochenendtagen.
Die Höhe der Säule einer Schülergruppe ergibt sich folglich aus der Summe der Mediane beider Kategorien.

% Analyse der Visualisierung
Der \ref*{fig:stacked_bar} kann entnommen werden, dass die Schülergruppen nahezu alle einen sehr geringen Alkoholkonsum an Werktagen aufweisen, und sich hauptsächlich durch ihren Konsum am Wochenende unterscheiden.
Eine Ausnahme bildet dabei die Gruppe, welche im Median beider Fächer die geringste Punktzahl von zwei bis vier aufweist. 
Hier liegt bereits der Konsum an Werktagen als einziges leicht über allen anderen Gruppen.
Nimmt man die zweite Kategorie mit in die Betrachtung auch zeigt dieselbe Gruppe erneut einen stärkeren Alkoholkonsum auf.
Insgesamt liegt die Gruppe mit der geringsten Medianpunktzahl mit 1.5 Einschätzungspunkten über allen anderen Gruppen, was einen negativen Effekt des Alkoholkonsums auf die Lernleistung vermuten lässt.
Unterstützt wird diese Vermutung durch die Untersuchung der weiteren Gruppen. 
Der Verlauf des Alkoholkonsums zeigt hier, dass der Konsum am Wochenende stufenweise ab einer Mediannote von 14 und ab einer Mediannote von 18 absinkt. 
Eine Ausnahme hierzu stellt die Gruppe im Bereich von vier bis sechs Mediannotenpunkten dar, dessen Konsum gleich dem der Gruppe mit der höchsten Mediannotenpunktzahl ist.

\begin{figure}[htb]
    \centering
    \includegraphics[width=1.0\textwidth]{src/visuals/image/violin.png}
    \caption{In der Literatur beschriebene Aspekte von datengesteuerten Organisationen}
    \label{fig:violin}
\end{figure}

% Vorstellung der Visualisierung
strukturelle Einflüsse auf die Lernleistung werden im Rahmen dieser Arbeit durch Betrachtung der Pendelzeit und des heimischen Internetzugangs untersucht.
Für die Untersuchung dient der in Abbildung \ref{fig:violin} dargelegte Violinen-Plot.
Dargestellt wird mit dieser Diagrammart die Datenverteilung der Variablen im Bezug auf die damit erreichten Notenpunktzahlen in Portugiesisch und Mathematik.
Hierbei legt die X-Achse die Pendelzeit anhand ihrer vier möglichen Werte (\textit{<15 min, 15-30 min, 30-60 min, >60 min}) und die Y-Achse die im Median erreichte Notenpunktzahl beider Fächer dar.
Eine \textit{Violine} wird dabei in der Verteilung linksseitig durch die der Gruppe des vorhandenen Internetzugangs und rechtsseitig durch die Gruppe des fehlenden Internetzugangs bestimmt.
Zu jeder Verteilung sind sowohl dessen Median als gestrichelte Linie und Ausreißer als Punkte erkennbar.
Die Farbe einer Verteilung orientiert sich hier anhand der Kategorie und folgt dem allgemeinen Farbverständnis von Grün als Bestätigung und Rot als Ablehnung des Internetzugangs.
% Analyse der Visualisierung


\section{Soziales Umfeld}

% c("school","sex","age","address","famsize","Pstatus","Medu","Fedu","Mjob","Fjob","reason","nursery","internet"))

% \item Erläuterung der Datenstruktur: merged, left join, individual (mat\_df, por\_df)
Der Einfluss des sozialen Umfelds eines Schülers auf dessen Lernleistung wird gleich der vorherigen Sektion aus fachübergreifender Perspektive betrachtet.
Dadurch kommen für die folgende Untersuchung die gleichen Datentransformationen zur Trage.

% \item Auswahl der Variablen in Abhängigkeit zur Frage:
%     \subitem address, famsize, Pstatus, medu, fedu, mjob, fjob, guardian, romantic, famrel, goout
In der Auswahl von Attributen wird sich zum einen auf das familiäre Umfeld beschränkt. 
Dabei werden konkret die Merkmale Familiengröße und Qualität der Familienbeziehungen untersucht, und die Information zum Zusammenleben der Eltern, Bildungsgrad und Arbeitsbereich der Eltern und dem Sorgerecht exkludiert.
Diese Entscheidung begründet sich darin, dass anzunehmen ist, dass sich die Effekte der ausgeschlossenen Variablen in der Qualität der Familienbeziehungen konsolidiert wiederspiegeln.
Zum anderen werden die Variablen romantic und goout der Analyse hinzugefügt, um das soziale Verhältnis mit gleichaltrigen zu beleuchten.
%     \subitem (famsize, famrel, G3)
%     \subitem (romantic, goout, G3)
% \item Grafik idee:
%     \subitem Scatterplot: Bestimmen von Personengruppen anhand von visuellem (1-2) oder maschinellem (3-x) Clustering.
%     \subitem Säulendiagramm: Median der Notenleistung (G3) zu den einzelnen Personengruppen

\begin{figure}[htb]
    \centering
    \includegraphics[width=1.0\textwidth]{src/visuals/image/bar.png}
    \caption{In der Literatur beschriebene Aspekte von datengesteuerten Organisationen}
    \label{fig:bar}
\end{figure}

\begin{figure}[htb]
    \centering
    \includegraphics[width=1.0\textwidth]{src/visuals/image/box.png}
    \caption{In der Literatur beschriebene Aspekte von datengesteuerten Organisationen}
    \label{fig:box}
\end{figure}

\section{Individuelle Leistungsbereitschaft}

% \item Erläuterung der Datenstruktur: merged, left join, individual (mat\_df, por\_df)
In der Untersuchung des Einflusses der individuellen Leistungsbereitschaft auf die Lernleistung werden die ursprünglichen fachspezifischen Datensätze verwendet.
Grund dafür ist der fehlende kausale Zusammenhang zwischen der Leistungsbereitschaft in einem Fach und der erzielten Lernleistung im anderen Fach.
Zu Berücksichtigen sind damit die unterschiedliche Datensatzgröße mit 395 (Mathematik-Schüler) und 649 (Portugiesisch-Schüler) Einträgen.
Entsprechend werden in dieser Sektion alle entwickelten Visualisierungen jeweils mit beiden unterschiedlichen Datengrundlagen dargestellt.

% \item Auswahl der Variablen in Abhängigkeit zur Frage:
    % \subitem studytime, failures, activities, schoolsup, famsup, paid, absences
Von den dem Handlungsbereich zugeordneten Attributen zeigen Lernzeit, Anzahl bereits durchgefallener Kurse, externe Lernunterstützung, familiäre Lernunterstützung, bezahlter Extraunterricht und Fehltage eine besondere Relevanz für die Lernleistung. 
In dieser Ergründung der Einflussfaktoren wird die AG-Teilnahme exkludiert, da hiermit zwar Zusatzarbeiten, jedoch abseits vom fachlichen Thema, geleistet werden.
Zusätzlich wird die Anzahl bereits durchgefallener Kurse nicht beachtet, weil hierbei kein kausaler Zusammenhang zur aktuellen Lernleistung besteht.

% \item Grafik idee:
    % \subitem (studytime, schoolsup, famsup, paid, absences)
    % \subitem Scatterplot: (Summe der Unterstützungen-schoolsup, famsup, paid) aggregierte Lernunterstützung (Summe der Unterstützungen) mit G3
    % \subitem Säulendiagramm oder Boxplots: (schoolsup, famsup, paid) Lernunterstützung  und deren aggr. median (G3)
    
    % \subitem (studytime, absences)
    % \subitem scatter3d 

\begin{figure}[htb]
    \centering
    \includegraphics[width=1.0\textwidth]{src/visuals/image/balk.png}
    \caption{In der Literatur beschriebene Aspekte von datengesteuerten Organisationen}
    \label{fig:balks}
\end{figure}


\begin{figure}[htb]
    \centering
    \includegraphics[width=1.0\textwidth]{src/visuals/image/nextbar.png}
    \caption{In der Literatur beschriebene Aspekte von datengesteuerten Organisationen}
    \label{fig:nextbar}
\end{figure}