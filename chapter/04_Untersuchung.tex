\chapter[Untersuchung der Forschungsfrage]{Untersuchung der Forschungsfrage}

Dieses Kapitel dient der Untersuchung der Forschungsfragen. 
In Abhängigkeit des untersuchten Handlungsbereichs werden die hierfür notwendigen Datentransformationen erläutert. 
Anschließend gilt es aufgrund der Vielzahl an Attributen, die für die Untersuchung relevanten Merkmale zu identifizieren. 
Final wird die Beantwortung der Forschungsfrage, durch die Vorstellung und Analyse der entwickelten Visualisierungen vorgenommen.

\section{Individuelle und strukturelle Faktoren}

% Erläuterung der Datenstruktur: merged, left join, individual (mat\_df, por\_df)
Zur Untersuchung dieser Faktoren eignet sich besonders ein fachübergreifende Perspektive, da eine fachspezifische unterschiedliche Auswirkung nur schwierig kausal zu erklären ist.
Im Weiteren wird dadurch eine Erhöhung der Datenmenge erzielt, da alle Attribute vollständig in dem Datensatz der Portugiesisch-Schüler und Mathematik-Schüler enthalten sind.
Die Migrierung beider Datenstände wird anhand der Attribute vorgenommen, welche im, unter (\cite[]{student_performance}) verfügbaren, R-Skript benannt werden.
Insgesamt erhält der Datensatz damit 657 Einträge. 

% \item Auswahl der Variablen in Abhängigkeit zur Frage:
% \subitem individuell: Geschlecht, Alter, Besuch der Vorschule, Absicht zur Weiterbildung, außerschulische Freizeit, Alkoholkonsum unter der Woche, Alkoholkonsum am Wochenende, Gesundheitszustand
% \subitem strukturell: Grund der Schulentscheidung, Pendelzeit, häuslicher Internetzugang,
Mit der in dieser Sektion betrachteten Faktoren ergibt sich die gleichnamige Aufteilung der Attribute zu den Aspekten der individuellen Eigenschaften und Verhaltensweisen und der strukturellen Gegebenheiten.
Individuelle Merkmale lassen sich zum einen durch die Möglichkeit zur aktuellen Einflussnahme durch den Schüler selbst filtern.
Zum Anderen tritt die Absicht zur Weiterbildung zwar in diesem Kontext als Prädiktor des Lernerfolgs auf, die alleinige Absicht selbst nimmt jedoch hierauf keinen unmittelbaren Einfluss.
Damit werden die Attribute \textit{außerschulische Freizeit, Alkoholkonsum unter der Woche, Alkoholkonsum am Wochenende, Gesundheitszustand} in die Untersuchung einfließen.
Strukturelle Gegebenheiten werden im Kontext dieser Arbeit durch den Grund der Schulentscheidung, die Pendelzeit und den häuslichen Internetzugang beschrieben.
Aus diesen Merkmalen kann besonders der Pendelzeit und dem häuslichen Internetzugang ein potenzieller kausaler Zusammenhang anhand des täglichen zusätzlichen Reiseaufwands und dem Zugang zu Onlinewissen unterstellt werden.
Daher werden dazu diese beiden Attribute für die Untersuchung ausgewählt.
% \item Grafik idee:
% \subitem : (außerschulische Freizeit, dalc, walc, health) 
% \subitem Scatterplot: (Pendelzeit, Internet, G3) strukturelle Anforderungen und Noteneffekt darstellen

% ~1100 WORDS UNTIL HERE !!! 
\section{Soziales Umfeld}

\begin{itemize}
    \item Erläuterung der Datenstruktur: merged, left join, individual (mat\_df, por\_df)
    \item Auswahl der Variablen in Abhängigkeit zur Frage:
    \subitem address, famsize, Pstatus, medu, fedu, mjob, fjob, guardian, romantic, famrel, goout
    \item Grafik idee:
    \subitem Scatterplot: Bestimmen von Personengruppen anhand von visuellem (1-2) oder maschinellem (3-x) Clustering.
    \subitem Säulendiagramm: Median der Notenleistung (G3) zu den einzelnen Personengruppen
\end{itemize}

\section{Individuelle Leistungsbereitschaft}

\begin{itemize}
    \item Erläuterung der Datenstruktur: merged, left join, individual (mat\_df, por\_df)
    \item Auswahl der Variablen in Abhängigkeit zur Frage:
    \subitem studytime, failures, activities, schoolsup, famsup, paid, absences
    \item Grafik idee:
\end{itemize}