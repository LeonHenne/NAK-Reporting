\chapter[Fazit und Ausblick]{Fazit und Ausblick}

% Answer research questions
Abschließend werden anhand dieses Kapitels die Ergebnisse der Untersuchungen zusammengefasst
In dieser Arbeit wurde sich der Zielstellung gewidmet die Beziehungen von Einflussfaktoren auf die schulische Leistung im Kontext der Oberschule zu untersuchen.
Dies wurde durchgeführt anhand der Forschungsfragen zur individueller Lernleistung, zum sozialem Umfeld und zu individuellen Verhaltensweisen und strukturellen Faktoren.
individuelle Verhaltensweisen wie der Konsum von Alkohol zeigten dabei einen negativ korrelierenden Effekt.
Ebenfalls wies der Datensatz eine Vermutung zu einem negativen Effekt steigender Pendelzeit und einem positiven Effekt von Internetzugang auf.
Das soziale Umfeld beeinflusste die Schüler des Datensatzes zum einen positiv mittels der Größe von intensiven Familienbeziehungen.
Zum anderen zeigte die Intensivierung sozialer Aktivitäten auf eine negativ korrelierende Lernleistung.
Die individuelle Leistungsbereitschaft wies einen unterstützenden Trend der Noten durch steigern der Lernzeit auf. 
Hingegen beschränkte sich der Effekt von Lernunterstützungen auf das Verdichten der Notenergebnisse.

% Limitierungen und Ausblick
Alle erläuterten Korrelationen werden in ihrer Aussagekraft limitiert durch die bewusste Beschränkung auf die final erreichte Note als Bemessung der Lernleistung.
Weiterhin zeigt das Forschungsfeld des Data Minings im Bildungswesen eine Vielzahl weiterer erhobener Daten auf.
Insgesamt führt dies dazu, dass es notwendig ist, alle in dieser Arbeit erkannten Phänomene und Korrelationen mittels zusätzlicher Daten zu validieren und auf Kausalität zu prüfen.
So kann erzielt werden, dass durch neue Erkenntnisse alle Zielgruppen in ihrer Aufgabe im Bildungssystem unterstützt und damit das Bildungsniveau angehoben werden kann.