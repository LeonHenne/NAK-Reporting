\chapter[Problemstellung]{Problemstellung}

Der für jede Nation erstrebenswerte langfristige ökonomische Fortschritt wird unter anderem stark durch das vorherrschende Bildungsniveau beeinflusst. \cite[S. 1]{Cortez2008UsingDM} 
Um dieses sich ergebende Bildungsniveau stärker zu durchleuchten und final zu verbessern, wird zur Unterstützung der Schüler und Lehrkräfte die Modellierung von Schulleistungen eingesetzt. \cite[S. 1]{Cortez2008UsingDM} 
So kann eine zeitabhängige Vorhersage der Leistungen, lernschwächere Schüler detektieren und damit Lehrkräfte frühzeitig befähigen, mit entsprechenden Maßnahmen in den Lernprozess einzugreifen. \cite[S. 2]{Namoun.2021} 
Verstärkt wurde dieser Bedarf durch die in der Vergangenheit eingetretenen Covid-Pandemie, und den damit verbundenen Schulschließungen, welche für neue erhebliche Herausforderungen sorgten. \cite[S. 2]{Clark.2021} 
Durch \cite[S. 13]{Clark.2021} konnte hierzu aufgezeigt werden, welchen positiven Effekt digitaler Lernunterstützung auf die Schülergruppen ausmachte.
Aus der Arbeit von \cite[S. 9]{Namoun.2021} geht jedoch hervor, dass bereits seit 2017 erneut das Interesse anstieg hinsichtlich der Modellierung von Lernergebnissen. 
Dabei wurde jedoch auch deutlich, dass seitdem besonders ein Fokus auf das Bildungsniveau von Bachelorstudiengängen besteht und die Untersuchung weiterführender Schulen lediglich ein Anteil von in etwa 12\% ausmacht. \cite[S. 11]{Namoun.2021} 
Die in den letzten Jahren erforschte Modellierung von Studierendenergebnissen lässt jedoch häufig unbeachtet, wie einzelne Faktoren innerhalb der maschinellen Lernmethoden zu den Vorhersagen führen. \cite[S. 19]{Namoun.2021}
Die Gesamtheit dieser aktuellen Gegebenheiten motiviert die nachfolgende Untersuchung des gewählten Datensatzes anhand der im Rahmen dieser Arbeit formulierten Forschungsfragen.