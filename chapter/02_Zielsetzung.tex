\chapter[Zielsetzung]{Zielsetzung}

Inhalt dieses Kapitels ist zunächst die Erläuterung des gewählten Datensatzes.
Zum Anderen werden daraus abgeleitete Forschungsfragen formuliert, welche im Rahmen dieser Arbeit durch die Entwicklung und Analyse von Visualisierungen untersucht werden. 

\section[Untersuchter Datensatz]{Datensatz}

Der in dieser Arbeit betrachtete Datensatz entstammt der Arbeit von \cite[]{Cortez2008UsingDM}.
Motiviert wurde diese Forschung durch Statistiken, welche Portugal im europäischen Vergleich als deutlich unterdurchschnittlich klassifizierten, aufgrund von hohen Durchfallquoten. \cite[S. 1]{Cortez2008UsingDM}
Daher wurde mit dieser Arbeit ein realer Datensatz erhoben.
Hierfür wurden Schulleisten und schulbezogene Faktoren vom Berichtswesen gesammelt und demografische und soziale Faktoren durch Befragungen ermittelt. \cite[S. 1]{Cortez2008UsingDM}
Die schulbezogenen Faktoren beziehen sich dabei auf die Leistungen in den Schulfächern Mathematik und Portugiesisch, da Inhalte dieser Fächer übergreifend in anderen Fächern zum Einsatz kommen. \cite[S. 2]{Cortez2008UsingDM}
Zielgruppe der Untersuchung waren Schüler der dreijährigen zweiten Bildungsphase in Portugal, welche auf der ersten neunjährigen Phase aufbaut. \cite[S. 2]{Cortez2008UsingDM}
Mittels der Berichte und Umfragen konnten schulbezogene-, demografische-, und soziale Faktoren von X Mathematik-Schülern und Y Portugiesisch-Schülern der beiden Schulen erhoben werden.
Alle erhobenen Faktoren, welche im Rahmen der Analyse auch als Variablen oder Merkmale bezeichnet werden, können der nachfolgenden Liste entnommen werden. 

%\begin{table}[!ht]
    \centering
    \begin{tabular}{lll}
    \hline
        \textbf{Beschreibung} & \textbf{Datentyp} & \textbf{Ausprägungen} \\ \hline
        Schulbezeichnung & Binär & GP - Gabriel Pereira; MS - Mousinho da Silveira \\ 
        Geschlecht & Binär & F - female; M - male \\ 
        Alter & Numerisch & 15 to 22 \\ 
        Wohngegend & Binär & U - urban; R - rural \\ 
        Familiengröße & Binär & LE3 - less or equal to 3; GT3 - greater than 3 \\ 
        Zusammenleben der Eltern & Binär & T - living together; A - apart \\ 
        Mutters Bildungsgrad & Numerisch & 0 - none; 1 - primary education (4th grade); 2 – 5th to 9th grade; 3 – secondary education; 4 – higher education \\ 
        Vaters Bildungsgrad & Numerisch & 0 - none; 1 - primary education (4th grade); 2 – 5th to 9th grade; 3 – secondary education; 4 – higher education \\ 
        Mutters Arbeitsbereich & Nominal & teacher; health care related; civil services (administrative or police); at home; other \\ 
        Vaters Arbeitsbereich & Nominal & teacher; health care related; civil services (administrative or police); at home; other \\ 
        Grund der Schulentscheidung & Nominal & close to home; school reputation; course preference; other \\ 
        Erziehungsberechtigter & Nominal & mother; father; other \\ 
        Pendelzeit & Numerisch & 1 - <15 min.; 2 - 15 to 30 min.; 3 - 30 min. to 1 hour; 4 - >1 hour \\ 
        Wöchentliche Lernzeit & Numerisch & 1 - <2 hours; 2 - 2 to 5 hours; 3 - 5 to 10 hours; 4 - >10 hours \\ 
        Anzahl durchgefallener Kurse & Numerisch & 1 to 3; 4 \\ 
        externe Lernunterstützung & Binär & yes; no \\ 
        familiäre Lernunterstützung & Binär & yes; no \\ 
        Extra bezahlter Unterricht im befragungsfach & Binär & yes; no \\ 
        außerschulische Aktivitäten & Binär & yes; no \\ 
        Besuch der Vorschule & Binär & yes; no \\ 
        Absicht zur Weiterbildung & Binär & yes; no \\
        häuslicher Internetzugang & Binär & yes; no \\
        Partnerliche Beziehung & Binär & yes; no \\
        Qualität der Familienbeziehungen & Numerisch & from 1 - very bad to 5 - excellent\\
        außerschulische Freizeit & Numerisch & from 1 - very low to 5 - very high \\
        soziale Aktivitäten & Numerisch & from 1 - very low to 5 - very high \\
        Alkoholkonsum unter der Woche & Numerisch & from 1 - very low to 5 - very high \\
        Alkoholkonsum am Wochenende & Numerisch & from 1 - very low to 5 - very high \\
        aktueller Gesundheitszustand & Numerisch & from 1 - very bad to 5 - very good\\
        Anzahl der Fehltage & Numerisch & from 0 to 93 \\
        erste Vorabnote & Numerisch & from 0 to 20 \\
        zweite Vorabnote & Numerisch & from 0 to 20 \\
        finale Note & Numerisch & from 0 to 20 \\
    \hline
    \end{tabular}
\end{table}


\begin{table}[htb]
    \centering
    \begin{tabular}{|l|l|l|}
    \hline
        Numerisch & \makecell[l]{Alter; Mutters Bildungsgrad; Vaters Bildungsgrad; Pendelzeit; \\Wöchentliche Lernzeit; Anzahl durchgefallener Kurse; \\Qualität der Familienbeziehungen; außerschulische Freizeit; \\soziale Aktivitäten; Alkoholkonsum unter der Woche; \\Alkoholkonsum am Wochenende; \\aktueller Gesundheitszustand; Anzahl der Fehltage; \\\textbf{erste Vorabnote; zweite Vorabnote; finale Note}} \\ \hline
        Binär & \makecell[l]{Schulbezeichnung; Geschlecht; Wohngegend; Familiengröße; \\Zusammenleben der Eltern; externe Lernunterstützung; \\familiäre Lernunterstützung; \\Extra bezahlter Unterricht im Befragungsfach; \\außerschulische Aktivitäten; Besuch der Vorschule; \\Absicht zur Weiterbildung; häuslicher Internetzugang; \\Partnerliche Beziehung} \\ \hline
        Nominal & \makecell[l]{Mutters Arbeitsbereich; Vaters Arbeitsbereich; \\Grund der Schulentscheidung; Erziehungsberechtigter} \\ \hline
    \end{tabular}
\end{table}

Detaillierte Informationen zu dessen Erläuterung und ihren Ausprägungen der \textbf{Tabelle X} im Anhang entnommen werden.

\section[Forschungsfragen]{Forschungsfragen}

\begin{itemize}
    \item Intro zur Absicht dieser Sektion
    \item Erläuterung der Absicht mit Bezug auf die Zielgruppe dieser Arbeit
    \item Einleitung zu den bereits erkannten Variablen ?
    \item Ableitung von interessanten Forsch
\end{itemize}