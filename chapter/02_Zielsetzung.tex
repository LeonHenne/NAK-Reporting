\chapter[Zielsetzung]{Zielsetzung}

Aus der dargelegten Problemstellung leitet sich für diese Arbeit die folgende Zielsetzung ab:

\begin{center}
    \textit{visuelle Untersuchung der Beziehungen von Einflussfaktoren auf die schulische Leistung im Kontext der Oberschule.}
\end{center}    

% Erläuterung der Absicht mit Bezug auf die Zielgruppe dieser Arbeit
Zielgruppe der Visualisierungen bilden Lehrerinnen und Lehrer sowie auch Sozialkräfte in schulischen Einrichtungen. 
Diese tragen u. a. die direkte Lehr- und Integrationsverantwortung, wobei davon auszugehen ist, dass sie durch ein besseres Verständnis der Lerneinflussfaktoren, darin unterstützt werden.
Aus dem organisatorischen- und Projektplanungsinteresse kann auch die Schulleitung der Zielgruppe hinzugefügt werden.
Angenommen wird, dass die Schulleitung mit einem besseren Verständnis der Einflussfaktoren eine effektivere Planung und Umsetzung von Schulgestaltungsprojekten, durchführen kann.

Trotz des hohen Domänenwissens verfügen damit nur wenige Persona der Zielgruppe über einen Hintergrund in Statistik oder Business Intelligence.
Dies ist bei der Entwicklung von Visualisierungen zu berücksichtigen, indem die Komplexität durch bspw. leicht lesbare Diagramme begrenzt wird.
% Dies ist bei der Entwicklung von Visualisierungen zu berücksichtigen, indem die Komplexität durch bspw. einfache Skalen und Diagrammtypen begrenzt wird.

% Zu dessen Bearbeitung wird in diesem Kapitel zunächst der hierfür gewählte Datensatz erläutert.
% Anschließend werden daraus abgeleitete Forschungsfragen formuliert, welche durch die Entwicklung und Analyse von Visualisierungen untersucht werden. 

\section[Untersuchter Datensatz]{Datensatz}

%  Intro zur Absicht dieser Sektion
Der in dieser Arbeit betrachtete Datensatz entstammt der Arbeit von (\cite[]{Cortez2008UsingDM}).
Motiviert wurde diese Forschung durch Statistiken, welche Portugal im europäischen Vergleich als deutlich unterdurchschnittlich klassifizierten, aufgrund von hohen Durchfallquoten (\cite[S. 1]{Cortez2008UsingDM}).
Daher wurde mit dieser Arbeit ein realer Datensatz erhoben.
Hierfür wurden Schulleisten und schulbezogene Faktoren vom Berichtswesen gesammelt und demografische und soziale Faktoren durch Befragungen ermittelt (\cite[S. 1]{Cortez2008UsingDM}).
Die schulbezogenen Faktoren beziehen sich dabei auf die Leistungen in den Schulfächern Mathematik und Portugiesisch, da Inhalte dieser Fächer übergreifend in anderen Fächern zum Einsatz kommen (\cite[S. 2]{Cortez2008UsingDM}).
Zielgruppe der Untersuchung waren Schüler der dreijährigen zweiten Bildungsphase in Portugal, welche auf der ersten neunjährigen Phase aufbaut (\cite[S. 2]{Cortez2008UsingDM}).
Mittels der Berichte und Umfragen wurden schulbezogene-, demografische-, und soziale Faktoren von 395 Mathematik-Schülern und 649 Portugiesisch-Schülern der \textit{Gabriel Pereira} und der \textit{Mousinho da Silveira} erhoben.
Alle erhobenen Faktoren, welche im Rahmen der Analyse auch als Variablen oder Merkmale bezeichnet werden, lassen sich mit ihrem Datentyp der \autoref{tab:Table2.1} entnehmen. 
Detaillierte Informationen zu dessen Erläuterung und ihren Ausprägungen der \autoref{tab:detailed_data} im Anhang entnommen werden.

\begin{table}[htb]
    \centering
    \caption{Kurzbeschreibung der im Datensatz enthaltenen Merkmale}
    \begin{tabular}{|l|l|l|}
    \hline
        Numerisch & \makecell[l]{Alter; Mutters Bildungsgrad; Vaters Bildungsgrad; Pendelzeit; \\Wöchentliche Lernzeit; Anzahl durchgefallener Kurse; \\Qualität der Familienbeziehungen; außerschulische Freizeit; \\soziale Aktivitäten; Alkoholkonsum unter der Woche; \\Alkoholkonsum am Wochenende; \\aktueller Gesundheitszustand; Anzahl der Fehltage; \\\textbf{erste Vorabnote; zweite Vorabnote; finale Note}} \\ \hline
        Binär & \makecell[l]{Schulbezeichnung; Geschlecht; Wohngegend; Familiengröße; \\Zusammenleben der Eltern; externe Lernunterstützung; \\familiäre Lernunterstützung; \\Extra bezahlter Unterricht im Befragungsfach; \\außerschulische Aktivitäten; Besuch der Vorschule; \\Absicht zur Weiterbildung; häuslicher Internetzugang; \\Partnerliche Beziehung} \\ \hline
        Nominal & \makecell[l]{Mutters Arbeitsbereich; Vaters Arbeitsbereich; \\Grund der Schulentscheidung; Erziehungsberechtigter} \\ \hline
    \end{tabular}
\end{table}

% TODO: Evtl: Datenerhebungsprozess

\section[Forschungsfragen]{Forschungsfragen}

% Ableitung von interessanten Forschungsfragen
Im Anschluss an die Beschreibung des Datensatzes gilt es die zuvor benannte Zielstellung zu konkretisieren. 
Dazu werden nachfolgend die für den Rahmen dieser Arbeit zu betrachteten Fragestellungen bestimmt, anhand derer in den folgenden Kapiteln Visualisierungen erstellt und analysiert werden.
Beziehen wir hierfür die verschiedenen Interessen und Handlungsmöglichkeiten der Zielgruppen ein, ergeben sich drei Haupthandlungsfelder für die Einflussnahme auf die Lernleistung.
Diese Umfassen die infrastrukturelle Gestaltung der Schule, die individuelle Förderung des sozialen Umfelds der Schüler und die unmittelbare Aufbereitung der Lehrinhalte als Unterrichtsformat.
Werden diesen Handlungsfeldern die aus dem Datensatz bekannten Attribute zugeordnet, ergeben sich folgende Forschungsfragen:
% Der Effekt dieser Handlungsfelder lässt sich mittels der folgenden Forschungsfragen untersuchen:

\begin{itemize}
    % structural
    \item Forschungsfrage 1: Welche individuellen und strukturellen Faktoren wirken sich auf die schulischen Leistungen aus ?
        % sex, age, reason, traveltime, internet, nursery, higher, freetime, dalc, walc, health
        \subitem \textbf{Attribute}: Geschlecht, Alter, Grund der Schulentscheidung, Pendelzeit, häuslicher Internetzugang, Besuch der Vorschule, Absicht zur Weiterbildung, außerschulische Freizeit, Alkoholkonsum unter der Woche, Alkoholkonsum am Wochenende, Gesundheitszustand
    % social
    \item Forschungsfrage 2: Wie beeinflusst das soziale Umfeld die Schulleistungen der Schülerinnen und Schüler ? 
        % address, famsize, Pstatus, medu, fedu, mjob, fjob, guardian, romantic, famrel, goout
        \subitem \textbf{Attribute}: Wohngegend, Familiengröße, Zusammenleben der Eltern, Mutters Bildungsgrad, Vaters Bildungsgrad, Mutters Arbeitsbereich, Vaters Arbeitsbereich, Erziehungsberechtigter, Partnerliche Beziehung, Qualität der Familienbeziehungen, soziale Aktivitäten
    % performance
    \item Forschungsfrage 3: Wie wird die tatsächliche Bewertung der Lernleistung von der individuellen Leistungsbereitschaft beeinflusst?
        % studytime, failures, activities, schoolsup, famsup, paid, absences
        \subitem \textbf{Attribute}: Lernzeit, Anzahl durchgefallener Kurse, AG-Teilnahme, externe Lernunterstützung, familiäre Lernunterstützung, bezahlter Extraunterricht, Fehltage
\end{itemize}

% bereits erkannte wichtige Variablen, bzw Annahmen zu den Variablen jedes Themenbereichs ?


% Formulierung aus FEDS 2023:
% die drei Primärziele dieser Studie dar. 
% Entsprechend der gängigen wissenschaftlichen Konvention, sind (nicht direkt validierbare) Primärziele noch in Sekundärziele aufzuteilen, 
% welche studienbezogen dann die Rolle der Akzeptanzkriterien einnehmen. 

% Die Primärziele gelten in diesem Sinne als erfüllt, wenn die jeweilig untergeordneten Sekundärziele nachweislich und vollständig erfüllt sind. 
% Wie der Nachweis hierzu erfolgt, ist Gegenstand von Kapitel 3, welches das wissenschaftliche Vorgehen im Rahmen dieser Studie näher erläutert.