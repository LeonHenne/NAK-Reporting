\chapter[Zielsetzung]{Zielsetzung}

Inhalt dieses Kapitels ist zunächst die Erläuterung des gewählten Datensatzes.
Zum Anderen werden daraus abgeleitete Forschungsfragen formuliert, welche im Rahmen dieser Arbeit durch die Entwicklung und Analyse von Visualisierungen untersucht werden. 

\section[Untersuchter Datensatz]{Datensatz}

Der in dieser Arbeit betrachtete Datensatz entstammt der Arbeit von \cite[]{Cortez2008UsingDM}.
Motiviert wurde diese Forschung durch Statistiken, welche Portugal im europäischen Vergleich als deutlich unterdurchschnittlich klassifizierten, aufgrund von hohen Durchfallquoten. \cite[S. 1]{Cortez2008UsingDM}
Daher wurde mit dieser Arbeit ein realer Datensatz erhoben.
Hierfür wurden Schulleisten und schulbezogene Faktoren vom Berichtswesen gesammelt und demografische und soziale Faktoren durch Befragungen ermittelt. \cite[S. 1]{Cortez2008UsingDM}
Die schulbezogenen Faktoren beziehen sich dabei auf die Leistungen in den Schulfächern Mathematik und Portugiesisch, da Inhalte dieser Fächer übergreifend in anderen Fächern zum Einsatz kommen. \cite[S. 2]{Cortez2008UsingDM}
Zielgruppe der Untersuchung waren Schüler der dreijährigen zweiten Bildungsphase in Portugal, welche auf der ersten neunjährigen Phase aufbaut. \cite[S. 2]{Cortez2008UsingDM}
Mittels der Berichte und Umfragen konnten schulbezogene-, demografische-, und soziale Faktoren von X Mathematik-Schülern und Y Portugiesisch-Schülern der beiden Schulen erhoben werden.
Alle erhobenen Faktoren, welche im Rahmen der Analyse auch als Variablen oder Merkmale bezeichnet werden, können mit ihrer Kurzbeschreibung und ihren Ausprägungen der nachfolgenden Liste entnommen werden:

\begin{table}[!ht]
    \centering
    \caption{Detaillierte Darstellung des untersuchten Datensatzes}
    \label{tab:detailed_data}
    \begin{tabular}{lll}
    \hline
        \textbf{Beschreibung} & \textbf{Datentyp} & \textbf{Ausprägungen} \\ \hline \hline
        Schulbezeichnung & Binär & \makecell[l]{GP - Gabriel Pereira; \\ MS - Mousinho da Silveira} \\ \hline
        Geschlecht & Binär & F - weiblich; M - männlich \\  \hline
        Alter & Numerisch & 15 bis 22 \\  \hline
        Wohngegend & Binär & U - urban; R - ländlich \\  \hline
        Familiengröße & Binär & \makecell[l]{LE3 - kleiner oder gleich 3; \\ GT3 - mehr als 3} \\  \hline
        Zusammenleben der Eltern & Binär & \makecell[l]{T - leben gemeinsam; \\ A - leben auseinander} \\  \hline
        Mutters Bildungsgrad & Numerisch & \makecell[l]{0 - kein Bildungsgrad; \\ 1 - Grundschulabschluss (4. Klasse); \\2 – 5. bis 9. Klasse; \\3 – Oberstufenabschluss; \\4 – höherer Bildungsgrad }\\  \hline
        Vaters Bildungsgrad & Numerisch & \makecell[l]{0 - kein Bildungsgrad; \\ 1 - Grundschulabschluss (4. Klasse); \\2 – 5. bis 9. Klasse; \\3 – Oberstufenabschluss; \\4 – höherer Bildungsgrad }\\  \hline
        Mutters Arbeitsbereich & Nominal & \makecell[l]{Lehrerin; Gesundheitswesen; \\Sozialwesen (Verwaltung oder Polizei); \\zuhause; anderer Bereich }\\  \hline
        Vaters Arbeitsbereich & Nominal & \makecell[l]{Lehrer; Gesundheitswesen; \\Sozialwesen (Verwaltung oder Polizei); \\zuhause; anderer Bereich }\\  \hline
        Grund der Schulentscheidung & Nominal & \makecell[l]{Nahe dem Zuhause; Ruf der Schule; \\Kurspräferenz; anderer Grund }\\  \hline
        Erziehungsberechtigter & Nominal & Mutter; Vater; Anderer \\  \hline
        Pendelzeit & Numerisch & \makecell[l]{1 - <15 min.; 2 - 15 bis 30 min.; \\3 - 30 min. bis 1 Stunde; 4 - >1 Stunde} \\  \hline
        Lernzeit & Numerisch & \makecell[l]{1 - <2 Stunden; 2 - 2 bis 5 Stunden; \\3 - 5 bis 10 Stunden; 4 - >10 Stunden }\\  \hline
        Anzahl durchgefallener Kurse & Numerisch & 1 bis 3; 4 \\  \hline
        externe Lernunterstützung & Binär & Ja; Nein \\  \hline
        familiäre Lernunterstützung & Binär & Ja; Nein \\  \hline
        bezahlter Extraunterricht & Binär & Ja; Nein \\  \hline
        AG-Teilnahme & Binär & Ja; Nein \\  \hline
        Besuch der Vorschule & Binär & Ja; Nein \\  \hline
        Absicht zur Weiterbildung & Binär & Ja; Nein \\ \hline
        häuslicher Internetzugang & Binär & Ja; Nein \\ \hline
        Partnerliche Beziehung & Binär & Ja; Nein \\ \hline
        Qualität der Familienbeziehungen & Numerisch & von 1 - sehr schlecht bis 5 - exzellent\\ \hline
        außerschulische Freizeit & Numerisch & von 1 - sehr schlecht bis 5 - sehr gut \\ \hline
        soziale Aktivitäten & Numerisch & von 1 - sehr schlecht bis 5 - sehr gut \\ \hline
        Alkoholkonsum unter der Woche & Numerisch & von 1 - sehr schlecht bis 5 - sehr gut \\ \hline
        Alkoholkonsum am Wochenende & Numerisch & von 1 - sehr schlecht bis 5 - sehr gut \\ \hline
        Gesundheitszustand & Numerisch & von 1 - sehr schlecht bis 5 - sehr gut\\ \hline
        Fehltage & Numerisch & von 0 bis 93 \\ \hline
        erste Vorabnote & Numerisch & von 0 bis 20 \\ \hline
        zweite Vorabnote & Numerisch & von 0 bis 20 \\ \hline
        finale Note & Numerisch & von 0 bis 20 \\ \hline
    \hline
    \end{tabular}
\end{table}



\section[Forschungsfragen]{Forschungsfragen}