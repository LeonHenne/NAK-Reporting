\chapter[Zielsetzung]{Zielsetzung}

Inhalt dieses Kapitels ist zunächst die Erläuterung des gewählten Datensatzes.
Anschließend werden daraus abgeleitete Forschungsfragen formuliert, welche im Rahmen dieser Arbeit durch die Entwicklung und Analyse von Visualisierungen untersucht werden. 

\section[Untersuchter Datensatz]{Datensatz}

Der in dieser Arbeit betrachtete Datensatz entstammt der Arbeit von \cite[]{Cortez2008UsingDM}.
Motiviert wurde diese Forschung durch Statistiken, welche Portugal im europäischen Vergleich als deutlich unterdurchschnittlich klassifizierten, aufgrund von hohen Durchfallquoten. \cite[S. 1]{Cortez2008UsingDM}
Daher wurde mit dieser Arbeit ein realer Datensatz erhoben.
Hierfür wurden Schulleisten und schulbezogene Faktoren vom Berichtswesen gesammelt und demografische und soziale Faktoren durch Befragungen ermittelt. \cite[S. 1]{Cortez2008UsingDM}
Die schulbezogenen Faktoren beziehen sich dabei auf die Leistungen in den Schulfächern Mathematik und Portugiesisch, da Inhalte dieser Fächer übergreifend in anderen Fächern zum Einsatz kommen. \cite[S. 2]{Cortez2008UsingDM}
Zielgruppe der Untersuchung waren Schüler der dreijährigen zweiten Bildungsphase in Portugal, welche auf der ersten neunjährigen Phase aufbaut. \cite[S. 2]{Cortez2008UsingDM}
Mittels der Berichte und Umfragen wurden schulbezogene-, demografische-, und soziale Faktoren von 395 Mathematik-Schülern und 649 Portugiesisch-Schülern der \textit{Gabriel Pereira} und der \textit{Mousinho da Silveira} erhoben.
Alle erhobenen Faktoren, welche im Rahmen der Analyse auch als Variablen oder Merkmale bezeichnet werden, können der nachfolgenden Liste entnommen werden. 

\begin{table}[htb]
    \centering
    \caption{Kurzbeschreibung der im Datensatz enthaltenen Merkmale}
    \begin{tabular}{|l|l|l|}
    \hline
        Numerisch & \makecell[l]{Alter; Mutters Bildungsgrad; Vaters Bildungsgrad; Pendelzeit; \\Wöchentliche Lernzeit; Anzahl durchgefallener Kurse; \\Qualität der Familienbeziehungen; außerschulische Freizeit; \\soziale Aktivitäten; Alkoholkonsum unter der Woche; \\Alkoholkonsum am Wochenende; \\aktueller Gesundheitszustand; Anzahl der Fehltage; \\\textbf{erste Vorabnote; zweite Vorabnote; finale Note}} \\ \hline
        Binär & \makecell[l]{Schulbezeichnung; Geschlecht; Wohngegend; Familiengröße; \\Zusammenleben der Eltern; externe Lernunterstützung; \\familiäre Lernunterstützung; \\Extra bezahlter Unterricht im Befragungsfach; \\außerschulische Aktivitäten; Besuch der Vorschule; \\Absicht zur Weiterbildung; häuslicher Internetzugang; \\Partnerliche Beziehung} \\ \hline
        Nominal & \makecell[l]{Mutters Arbeitsbereich; Vaters Arbeitsbereich; \\Grund der Schulentscheidung; Erziehungsberechtigter} \\ \hline
    \end{tabular}
\end{table}

Detaillierte Informationen zu dessen Erläuterung und ihren Ausprägungen der \textbf{Tabelle X} im Anhang entnommen werden.

\textbf{Evtl: Datenerhebungsprozess}

\section[Forschungsfragen]{Forschungsfragen}

\begin{itemize}
    \item Intro zur Absicht dieser Sektion
    \item Erläuterung der Absicht mit Bezug auf die Zielgruppe dieser Arbeit
    \item Einleitung zu den bereits erkannten Variablen ?
    \item Ableitung von interessanten Forsch
\end{itemize}